
%--------------------------------------------------------------------------------------------------------------------  
%--------------------------------------------------------------------------------------------------------------------

  \subsection{Finalidad del proyecto (a quien ayuda, para que sirve)}  

 En el laboratorio de sólidos amorfos de la facultad de ingeniería de la universidad de Buenos Aires funciona el 
 INTECIN y en el mismo el \href{''http://intecin.fi.uba.ar/grupos.php?grupo=12''}{grupo de materiales nanoestructurales}, este grupo prepara y
 estudia sistemas basados en nanopartículas magnéticas apuntando a las posibles aplicaciones tecnológicas
 en sensores, remediación ambiental y aplicaciones clínicas. Sus integrantes tienen experiencia en caracterización estructural por difracción, dispersión y absorción
 de rayos X, espectroscopia Mössbauer, propiedades magnéticas y de magnetotransporte y microscopía electrónica.\newline
  
  Para lograr esto se realiza el sintetizado de nano estructuras, obteniendo una cinta metálica con una estructura amorfa (sin estructura cristalina)
  similar a un vidrio. Mediante un proceso de pulverizado de esta cinta se obtiene un polvo cuyas partículas poseen una estructura nanométrica.
  Para poder obtener una estructura sólida a partir del polvo se utiliza el proceso de sinterizado. Este proceso
  logra unificar las particulas del polvo en una sola estructura sólida. Para este fin se necesita que una intensidad de corriente
  elevada, en el orden de los KA, atraviese la muestra de polvo nonometrico. Esta corriente puede ser provista por un banco de 
  capacitores (proceso rápido) o una fuente de RF (proceso lento).\newline

  La finalidad de nuestro proyecto es desarrollar una automatización y relevamiento del proceso de sinterizado a partir de la 
  descarga de un banco de capacitores, realizar recopilación de datos del proceso con la posibilidad de variar alguna variables
  involucradas en el sitenrizado para evaluar el desempeño del mismo y sus resultados.

  \newpage


%--------------------------------------------------------------------------------------------------------------------  
%--------------------------------------------------------------------------------------------------------------------

  \subsection{Planteamiento del problema a resolver}  

  Para el estudio y desarrollo de cualquier tecnología es indispensable la experimentación, ya que es el medio atraves del cual
  se puede comprender y validar los desarrollos teóricos del proceso estudiado. La flexibilidad a la hora de la experimentación
  es factor deseado, entendiendo por flexibilidad la posibilidad de poder modificar parámetros del experimento de forma rapida y
  sencilla, ya que facilita y ahora tiempos a la hora del desarrollo del conocimiento. La fácil obtención y disponibilidad de 
  los resultados de la experimentación es otro deseado en la experimentación. \newline

  Para poder estudiar y obtener materiales nanometricos atraves del proceso de sinterizado es necesario contar con un plataforma 
  que controle y monitoree el proceso de sinterizado de forma flexible y parametrica. La plataforma le debe brindar al investigador
  la posibilidad de modificar las variables del proceso y obtener los resultados de la experimentacion.
  El proceso de sinterizado requiere la manipulación de una potencia electrica conciderable, es por ello que la plataforma debe	
  brindar un manejo seguro de esta potencia , contando con todas medidas de seguridad requeridas para asegurar una segura operación.

  \newpage

%--------------------------------------------------------------------------------------------------------------------  
%--------------------------------------------------------------------------------------------------------------------
