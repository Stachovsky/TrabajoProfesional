%--------------------------------------------------------------------------------------------------------------------  
%--------------------------------------------------------------------------------------------------------------------

  \subsection{Historia y antecedentes}  

Los materiales nanocompuestos formados en su totalidad por particulas cerámicas y metálicas en el rango de los centenares
  y decenas de nanómetros, son una clase de materiales que presentan propiedades físicas muy distíntas a las de esos mismos
 con tamaños de cristal micrométricos (materiales convencionales). Tales propiedades, al ampliar su rango de funcionalidad
 resultan muy interesantes desde el punto de vista tecnológico ya que permiten combinar propiedades excepcionales con funcionalidad
 imprescindibles en aplicaciones específicas, como la transparencia, la biocompatibilidad, la dureza o conductividad eléctrica, entre otras.

 Sin embargo, la aplicacion industrial de los materiales nanocompuestos radica en la capacidad de consolidación de estos materiales
 formando cuerpos densos y compuestos, pero preservando el tamañan nanométrico de sus componentes. Las técnicas de consolidación convencionales
 presentan importantes limitaciones y no son capaces de preservar esta nanoestructura. 
 La solución de estos problemas y el desarrollo de nuevas tecnologías ha posibilitado la producción de materiales nanoestructurales.

  \newpage

%--------------------------------------------------------------------------------------------------------------------  
%--------------------------------------------------------------------------------------------------------------------

  \subsection{Definiciones y glosario de términos}  

  \newpage


%--------------------------------------------------------------------------------------------------------------------  
%--------------------------------------------------------------------------------------------------------------------

  \subsection{Justificación del proyecto}  

  \newpage

%--------------------------------------------------------------------------------------------------------------------  
%--------------------------------------------------------------------------------------------------------------------

