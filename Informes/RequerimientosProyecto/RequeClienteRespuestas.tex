\documentclass[12pt]{article}

\usepackage{graphicx}
\usepackage{caption}
\usepackage{subcaption}
\usepackage[spanish]{babel}
\usepackage[utf8]{inputenx}
\usepackage{anysize} 
\usepackage{amsmath}
\usepackage{listings}
%\usepackage{wrapfig} % paquete gráfico... quizás es necesario actualizar paquetes en kile
\usepackage{hyperref}

\usepackage{fancyhdr}
\pagestyle{fancy}

\hypersetup{
    colorlinks,%
    citecolor=black,%
    filecolor=black,%
    linkcolor=black,%
    urlcolor=black
}
\marginsize{1cm}{1cm}{1cm}{1cm} %izq der arriva abajo

\title{Informe preliminar}
\author{Facundo Nahuel Uriel Silva}

\lhead{ Anexo 1 }
 
\begin{document}

\section{Anexo}

\section{Requisitos Técnicos}

\subsection{Preguntas}
  
 \begin{enumerate} 
      \item \textbf{ Proceso de sinterizado }
	\begin{enumerate}
	  \item \textbf{ ¿Que magnitudes físicas se deben medir? }
		\subitem	\textit{ Las magnitudes que el sistema de debe medir son: corriente que circula en el momento de la descarga y 
					tensión sobre la muestra. Se evaluará la posibilidad de medir temperatura sobre la muestra de existir un
					sensor de temperatura con una respuesta temporal adecuada.
			}

	  \item \textbf{ ¿Cual es el mínima corriente requerida para el proceso? }
		\subitem	\textit{ La importancia radica en la corriente de pico al momento de la descarga.
				}

	  \item \textbf{ ¿Cual es el máxima corriente de pico esperada? }
		\subitem	\textit{ La magnitud del pico de corriente eléctrica en la descarga estará comprendido entre 1KA y 10KA.
				}

	  \item \textbf{ ¿Cual es el orden de magnitud de la impedancia eléctrica de la muestra de polvos? }
		\subitem	\textit{ La muestra estará constituida por un material metálico por lo que su impedancia estará en el orden los los m$\Omega$.
				}

	  \item \textbf{ ¿Se experimentará con distintos tipos de polvos? }
		\subitem	\textit{ No, solo se utilizaran polvos de materiales metálicos.
				}

          \item \textbf{ ¿Qué materiales en particular se van utilizar como muestra a sinterizar? }
		\subitem	\textit{ Se utilizaran materiales basados en hierro (Fe).
				}

          \item \textbf{ ¿Se proyecta que a futuro se utilice otros materiales?¿Cómo afectaría esto al proceso? }
		\subitem	\textit{ No, este tipo de tecnología solo se puede aplicar a materiales de tipo metálico.
				}

          \item \textbf{ ¿Existe algún proceso por el cual se puede determinar que la muestra está sinterizada correctamente?¿Se desea implementar? }
		\subitem	\textit{ El resultado del proceso se validará con una batería de pruebas y estudios externos al proceso
					  que se le realizaran a la muestra.
				}

	  \item \textbf{ ¿Existirá un único banco de capacitores (descargar múltiples secuenciales)? }
		\subitem	\textit{ El sistema debe poder ser escalable. Se debe especificar que cambios se deben hacer en caso de decir escalar
					la capacidad de corriente del sistema en los cables de conexión , sensores y  en los contactores del dispositivo.
				}

	  \item \textbf{ ¿Con qué periodicidad se estima realizar el proceso (horas, días, semanas)? }
		\subitem	\textit{ Se estima unos 60 usos por mes.
				}

	  \item \textbf{ ¿En cuanto a la compresión mecánica, qué prensa se utilizará? }
		\subitem	\textit{ Se utilizara un prensado manual o automático, pero en ambos casos el proceso de prensado será externo al dispositivo.
				}

	  \item \textbf{ ¿El valor de presión que se establece antes de empezar la descarga, debe reajustarse durante el proceso?¿Cual tiene que ser este valor constante? }
		\subitem	\textit{ No, la muestra del polvo a sinterizar se compactar antes de la descarga.
				}


	\end{enumerate}
	  
      \item \textbf{ Interfaz de usuario }
	\begin{enumerate}
	  \item \textbf{ ¿Cómo se desea visualizar los datos obtenidos del proceso? }
		\subitem	\textit{ Se desea tener una aplicación de PC dedicada al monitoreo, control y administración del sistema.
					 La aplicación debe poder funcionar en las plataformas Windows y Linux.
				}

	  \item \textbf{ ¿Es necesario un que el sistema tenga un display ?¿ Y teclado? }
		\subitem	\textit{ Si para la administración básica del dispositivo. La administración y gestion compleja del dispositivo se hará
					 a traves de la PC.
				}

	  \item \textbf{ ¿Se necesita accionar en forma manual algún parámetro del proceso, Cuál? }
		\subitem	\textit{ No, la totalidad del proceso de sinterizado y el monitoreo de las magnitudes deber ser automático.
				}

	  \item \textbf{ ¿En necesario visualizar los datos en forma remota (vía Web)? }
		\subitem	\textit{ No es necesario ya que no seria de utilidad.
				}

	  \item \textbf{ ¿Se cuenta con bocas de red cerca de la zona de emplazamiento del dispositivo?¿Se planea hacerlo? }
		\subitem	\textit{ No.
				} 

	  \item \textbf{ Los datos de la experimentación, ¿Deben quedar guardados en el dispositivo o un servidor local? }
		\subitem	\textit{ Si, los datos se almacenaran en el dispositivo hasta su descarga mediante el programa de PC.
				}

	  \item \textbf{ ¿Es necesario tener la posibilidad de guardar los datos en un pendrive? }
		\subitem	\textit{ Si es de utilidad.
				}

	  \item \textbf{ ¿Se desea genera alguna extensión de archivo en particular? }
		\subitem	\textit{ Es deseable el formato CSV.
				}

	  \item \textbf{ ¿Cómo desea configurar el sistema de control? }
		\subitem	\textit{ -------------
				}

	  \item \textbf{ ¿Que parámetros del proceso se deben visualizar y cuales almacenar (magnitudes física)? }
		\subitem	\textit{ Se mostraran el estado de los parámetros en el display pero los datos relevados de la experimentación se
					  obtendrán mediante el programa de PC.
				}

	  \item \textbf{ ¿Qué parámetros de control tiene el proceso (condiciones que se deben cumplir para iniciar el proceso. Ejemplo: nivel de carga)? }
		\subitem	\textit{ -------------
				}

	  \item \textbf{ ¿Qué parámetros de control deberían ser establecidos de forma remota y cuales de forma local? }
		\subitem	\textit{ La configuración del sistema será mediante el programa de PC.
				}

	  \item \textbf{ ¿Se necesita accionar en forma manual algún parámetro del proceso?¿Cuál? }
		\subitem	\textit{ No.
				}

	\end{enumerate}
      
      \item \textbf{ Seguridad }
	\begin{enumerate}
	  \item \textbf{ ¿Es necesario algún parámetro de seguridad en especial?¿Qué es lo más crítico del proceso? }
		\subitem	\textit{ Debe cumplir la normativa de seguridad impuesta en el laboratorio.
				}

	  \item \textbf{ ¿El operario estará en el mismo ambiente de la experimentación? }
		\subitem	\textit{ Si.
				}

	  \item \textbf{ ¿Debe haber elementos contra incendios? }
		\subitem	\textit{ No son necesarios.
				}

	  \item \textbf{ ¿Es necesario un nivel de autorización para operar el dispositivo (login)? }
		\subitem	\textit{ Si.
				}

	\end{enumerate}
      
    \end{enumerate}

\end{document}